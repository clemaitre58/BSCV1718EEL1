\documentclass[12pt]{tdtp}
\usepackage{tabularx,colortbl}
\usepackage{multirow}
\usepackage{listings}
\lstset{
	language=VHDL,
basicstyle=\tiny\ttfamily}
\definecolor{light-gray}{gray}{0.96}
\definecolor{pageheading-gray}{gray}{0.2}
\definecolor{dark-gray}{gray}{0.45}
\definecolor{dark-green}{rgb}{0.245,0.121,0.0}

\newcommand{\auteur}{Cedric Lemaitre}
\newcommand{\couriel}{c.lemaitre58@gmail.com}
\newcommand{\promo}{Bachelor in Computer Vision}
\newcommand{\annee}{2017-2018}
\newcommand{\matiere}{Digital Electronics}

\newcommand{\tdtp}{Labs 1}
\renewcommand{\sujet}{VHDL Design}


\begin{document}
\titre
We propose in this labs some small basic exercises wich allow to discover VHDL Language.\\

\textit {NB : all examples should be check using \textbf{test-benchs} and simulatuon tools}


%%%%%%%%%%%%
\Exo

Create a AND3 with :

\begin{enumerate}
	\item 3 inputs : a,b,c
	\item 1 output : s
	\item Concurrency assignement 
\end{enumerate}


%%%%%%%%%%%%
\Exo

Create a ADD1\footnote{Add 2 signals of 1 bit} with :

\begin{enumerate}
	\item 3 inputs : a, b, cin
	\item 2 output : s, cout
	\item Concurrency assignement 
\end{enumerate}


%%%%%%%%%%%
\Exo 

Create a ADD4\footnote{Add 2 signals of 4 bit} with : 

\begin{enumerate}
	\item 3 inputs : a, b, cin
	\item 2 outputs : s, cout
	\item Concurrency assignement 
	\item use your previous ADD1 as a function
\end{enumerate}

\textit{Please consider \textbf{port map} instructions for instantation process}

%%%%%%%%%%
\Exo


Create a D flipflop  with : 

\begin{enumerate}
	\item 2 inputs : D, clk
	\item 1 output : q
\end{enumerate}
%%%%%%%%%%%%%%%%%%%%%%%%%%%%%%%%%%

\Exo


Create an asynchronous 8-bit counter with : 

\begin{enumerate}
	\item 2 inputs : reset, clk
	\item 1 output : s
\end{enumerate}




%%%%%%%%%%
\Exo

Create an 8-bit shift register : \\

In digital circuits, a shift register is a cascade of D flip flops, sharing the same clock, in which the output of each flip-flop is connected to the 'data' input of the next flip-flop in the chain, resulting in a circuit that shifts by one position the 'bit array' stored in it, 'shifting in' the data present at its input and 'shifting out' the last bit in the array, at each transition of the clock input.



%%%%%%%%%

\Note

A test bench is an environment used to verify the correctness or soundness of a design or model, for example, that of a software product.
ISE software allows to create vhdl test bench files as follow :  

\begin{lstlisting}
library IEEE;
use IEEE.std_logic_1164.all;
use IEEE.numeric_std.all;

entity test is
end entity;

architecture arc of test is
  signal a, b : unsigned(3 downto 0);

  signal cout1, cout2 : std_logic;

  signal i, j, k      : integer := 0; 
  signal error_found1 : integer := 0;  
  signal error_found2 : integer := 0;   
  signal clk          : std_logic;  

  component add_struct is
    port
      (
        a, b : in  unsigned(3 downto 0);
        s    : out unsigned(3 downto 0);
        cout : out std_logic
        );
  end component;

  component add_comp is
    port (
      a, b : in  unsigned(3 downto 0);
      s    : out unsigned(3 downto 0);
      cout : out std_logic
      );
  end component;

begin

  inst1 : add_struct port map (a, b, s1, cout1);
  inst2 : add_comp port map (a, b, s2, cout2);



  process
  begin
    clk <= '0';
    wait for 5 ns;
    clk <= '1';
    wait for 5 ns;
  end process;


  a <= to_unsigned (i, 4);
  b <= to_unsigned (j, 4);
  k <= i+j;

  process
  begin
    wait until clk'event and clk = '1';
    if k /= to_integer(cout1 & s1) then
      error_found1 <= 1;
    end if;
    if k /= to_integer(cout2 & s2) then
      error_found2 <= 1;
    end if;

    if i = 15 then
      i <= 0;
      j <= j + 1;
    else
      i <= i + 1;
    end if;

    
  end process;




end arc;
\end{lstlisting}

Table \ref{vhdl_ref_words} provides a complete list of VHDL reserved words.

\begin{table}[!h]
	\centering
	\small\textsf{
		\begin{tabular*}{0.9\textwidth}{@{\extracolsep{\fill}} l l l l l }
			\hline
			abs            &  downto    &  library  &  postponed  &  srl        \\
			\rowcolor{light-gray} access         &  else      &  linkage  &  procedure  &  subtype    \\
			after          &  elsif     &  literal  &  process   &  then        \\
			\rowcolor{light-gray} alias          &  end       &  loop     &  pure      &  to          \\
			all            &  entity    &  map      &  range     &  transport   \\
			\rowcolor{light-gray} and            &  exit      &  mod      &  record    &  type        \\
			architecture   &  file      &  nand     &  register  &  unaffected  \\
			\rowcolor{light-gray} array          &  for       &  new      &  reject    &  units       \\
			assert         &  function  &  next     &  rem       &  until       \\
			\rowcolor{light-gray} attribute      &  generate  &  nor      &  report    &  use         \\
			begin          &  generic   &  not      &  return    &  variable    \\
			\rowcolor{light-gray} block          &  group     &  null     &  rol       &  wait        \\
			body           &  guarded   &  of       &  ror       &  when        \\
			\rowcolor{light-gray} buffer         &  if        &  on       &  select    &  while       \\
			bus            &  impure    &  open     &  severity  &  with        \\
			\rowcolor{light-gray} case           &  in        &  or       &  signal    &  xnor        \\
			component      &  inertial  &  others   &  shared    &  xor         \\
			\rowcolor{light-gray} configuration  &  inout     &  out      &  sla       &              \\
			constant       &  is        &  package  &  sll       &              \\
			\rowcolor{light-gray} disconnect     &  label     &  port     &  sra       &              \\
			\hline
	\end{tabular*}}
	\caption{A complete list of VHDL reserved words.}
	\label{vhdl_ref_words}
\end{table}

\end{document}
